%-----------------------------------------------------------------------------------------------%
%
% % Oktober 2022
% Template Latex untuk Laporan Kerja Praktek Program Studi Sistem informasi ini
% Dikembangkan oleh Daffa Takratama Savra (daffatakratama13@gmail.com)

% Template ini dikembangkan dari template yang dibuat oleh Inggih Permana (inggihjava@gmail.com).

% Orang yang cerdas adalah orang yang paling banyak mengingat kematian.
%
%-----------------------------------------------------------------------------------------------%
\fontsize{12}{14.4}
\begin{center}\MakeUppercase{\textbf{Abstrak}}\end{center}

\noindent
\fontsize{10pt}{12pt}\selectfont
Program Studi Sistem Informasi di Fakultas Sains dan Teknologi UIN Suska Riau telah lama menjadi pusat fokus dalam bidang Teknologi Informasi. Namun, manajemen inventaris di laboratorium seperti Laboratorium Rekayasa Sistem Informasi, Internet, dan \textit{Software Engineering} masih dilakukan secara manual, mengakibatkan inventaris tidak dikelola secara efisien. Untuk mengatasi masalah ini, SITARIS, sebuah sistem informasi inventaris, diimplementasikan dengan menggunakan \textit{framework} CodeIgniter 4. Kelebihan CodeIgniter dalam ukuran dan efisiensi menjadikannya pilihan utama. Penelitian ini bertujuan untuk mengimplementasikan sistem informasi inventaris yang terkomputerisasi dengan efektif dan efisien di lingkungan Laboratorium Program Studi Sistem Informasi. Melalui penerapan \textit{framework} CodeIgniter 4, pengelolaan barang di laboratorium dapat dipermudah. Manfaatnya termasuk efisiensi pengolahan data, pengelolaan inventaris yang lebih baik, akses cepat ke informasi inventaris, dokumentasi, pengelolaan peminjaman, serta pengambilan keputusan yang lebih baik dalam perawatan dan alokasi pendanaan. Dengan demikian, implementasi SITARIS dengan \textit{framework} CodeIgniter 4 memberikan manfaat signifikan bagi Laboratorium Program Studi Sistem Informasi.\\
\noindent{\textbf{Kata Kunci:} CodeIgniter, Inventaris Laboratorium, MariaDB, PHP, Sistem Informasi} \\