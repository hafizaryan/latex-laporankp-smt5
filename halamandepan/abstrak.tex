%-----------------------------------------------------------------------------------------------%
%
% % Oktober 2022
% Template Latex untuk Laporan Kerja Praktek Program Studi Sistem informasi ini
% Dikembangkan oleh Daffa Takratama Savra (daffatakratama13@gmail.com)

% Template ini dikembangkan dari template yang dibuat oleh Inggih Permana (inggihjava@gmail.com).

% Orang yang cerdas adalah orang yang paling banyak mengingat kematian.
%
%-----------------------------------------------------------------------------------------------%
\fontsize{12}{14.4}
\begin{center}\MakeUppercase{\textbf{Abstrak}}\end{center}

\noindent
\fontsize{10pt}{12pt}\selectfont
Program Studi Sistem Informasi di Fakultas Sains dan Teknologi UIN Suska Riau telah lama menjadi fokus utama dalam bidang Teknologi Informasi. Meskipun demikian, manajemen inventaris di beberapa laboratorium seperti Laboratorium Rekayasa Sistem Informasi, Internet, dan \textit{Software Engineering} masih dilakukan secara manual, mengakibatkan kurang efisiennya pengelolaan inventaris. Untuk mengatasi permasalahan ini, diterapkan SITARIS, suatu sistem informasi inventaris, menggunakan \textit{framework} CodeIgniter 4. Keunggulan CodeIgniter dalam hal ukuran dan efisiensi membuatnya menjadi pilihan utama. Tujuan dari penelitian ini adalah mengimplementasikan sistem informasi inventaris yang terkomputerisasi secara efektif dan efisien di lingkungan Laboratorium Program Studi Sistem Informasi. Dengan menerapkan \textit{framework} CodeIgniter 4, pengelolaan barang di laboratorium dapat dilakukan dengan lebih mudah. Manfaatnya mencakup efisiensi dalam pengolahan data, peningkatan pengelolaan inventaris, akses cepat ke informasi inventaris, dokumentasi yang baik, manajemen peminjaman yang terorganisir, serta pengambilan keputusan yang lebih baik terkait perawatan dan alokasi pendanaan. Dengan demikian, implementasi SITARIS menggunakan \textit{framework} CodeIgniter 4 memberikan manfaat yang signifikan bagi Laboratorium Program Studi Sistem Informasi.\\
\noindent{\textbf{Kata Kunci:} CodeIgniter, Inventaris Laboratorium, MariaDB, PHP, Sistem Informasi} \\