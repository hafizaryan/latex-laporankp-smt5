%-----------------------------------------------------------------------------------------------%
%
% % Oktober 2022
% Template Latex untuk Laporan Kerja Praktek Program Studi Sistem informasi ini
% Dikembangkan oleh Daffa Takratama Savra (daffatakratama13@gmail.com)

% Template ini dikembangkan dari template yang dibuat oleh Inggih Permana (inggihjava@gmail.com).

% Orang yang cerdas adalah orang yang paling banyak mengingat kematian.
%-----------------------------------------------------------------------------------------------%

%-----------------------------------------------------------------------------%
\chapter*{\kataPengantar}
%-----------------------------------------------------------------------------%
Pada kesempatan ini penulis mengucapkan puji syukur atas kehadirat Allah SWT, karena dengan Rahmat dan Karunia-Nya penulis dapat menyusun dan menyelesaikan Laporan Kerja Praktek ini yang berjudul “Implementasi Sistem Informasi Inventaris (SITARIS) Menggunakan \textit{Framework} CodeIgniter 4 (Studi kasus : Laboratorium Prodi Sistem Informasi UIN Suska Riau)”. Shalawat dan salam tidak lupa pula penulis ucapkan kepada Rasulullah Muhammad SAW, dengan mengucapkan “Allahumma Sholli Ala Saidina Muhammad, Wa’ala Alihi Saidina Muhammad”.

Penulisan dan penyusunan Laporan Kerja Praktek ini tidak terlepas dengan adanya bantuan dari berbagai pihak, baik yang berupa materi maupun berupa motivasi. Untuk itu pada kesempatan ini penulis mengucapkan banyak terima kasih kepada:

\begin{enumerate}
	\item Bapak \rektor., selaku Rektor \universitas.
	\item Bapak \dekan., selaku Dekan \fakultas.
	\item Bapak \kaprodi., selaku Ketua \programStudi \space
	      \fakultas \space \universitas.
	\item Ibu \sekretarisprodi., Sekretaris dan Koordinator Kerja Praktek \programStudi \space
	      \fakultas \space \universitas
	\item Bapak Tengku Khairil Ahsyar, S.Kom., M.Kom., Dosen Pembimbing Kerja Praktek Sekaligus Kepala Laboratorium Prodi Sistem Informasi yang telah berkenan membimbing dan meluangkan banyak waktu, tenaga dan pikiran guna mengarahkan penulis dalam menyelesaikan Laporan Kerja Praktek ini.
	\item Ibu Mona Fronita, S.Kom., M.Kom., selaku Pembimbing Akademik yang telah memberikan dukungan, arahan, dan masukan kepada penulis dari awal perkuliahan hingga saat ini.
	\item Segenap Dosen dan Karyawan Program Studi Sistem Informasi Fakultas Sains dan Teknologi Universitas Islam Negeri Sultan Syarif Kasim Riau.
	\item Kedua orang tua dan keluarga yang selalu memberikan semangat baik berupa moril maupun materil, motivasi dan doa setiap waktu.
	\item Terkhusus diri pribadi penulis karena telah berhasul melalui proses perkuliahan yang panjang serta tetap semangat dan kuat melewati proses pendewasaan diri.
	      % \item Rahma Yulia Fani yang telah memberikan \textit{support}, semangat dan motivasi kepada penulis dalam menyelesaikan Laporan Kerja Praktek ini.
	      % \item Terima kasih penulis ucapkan kepada Nasya Amirah Melyani dan Indah Nirwana selaku tim dalam proyek kerja praktek ini, dan juga terima kasih penulis ucapkan kepada Ahmad Dhani dan M. Wira selaku teman-teman dalam tim SIMLAB ini.
	\item Teman-teman semua, terkhusus sahabat-sahabat Sistem Informasi seluruh angkatan 2021, terima kasih atas bantuan dan motivasi kalian. Tetap semangat, selalu junjung kesabaran, keikhlasan dan kekompakan untuk kita semua.

\end{enumerate}

Semoga kebaikan yang telah diberikan kepada penulis mendapat balasan dan diterima oleh Allah SWT, Aamiin. Penulis menyadari bahwa penulisan laporan Kerja Praktek yang telah dibuat ini masih belum sempurna dan masih banyak kekurangan baik dari segi teknis maupun penyusunannya. Oleh karena itu, penulis menerima kritik dan saran yang membangun demi kesempurnaan laporan Kerja Praktek ini. Akhir kata, penulis ucapkan terima kasih.

\vspace*{0.1cm}

\begin{flushright}
	\kota, \tanggalPersetujuan\\
	Penulis,\\
	\vspace{2cm}
	\textbf{\underline{\penulis}\\
		NIM. \nim}

\end{flushright}

