%-----------------------------------------------------------------------------------------------%
%
% % Oktober 2022
% Template Latex untuk Laporan Kerja Praktek Program Studi Sistem informasi ini
% Dikembangkan oleh Daffa Takratama Savra (daffatakratama13@gmail.com)

% Template ini dikembangkan dari template yang dibuat oleh Inggih Permana (inggihjava@gmail.com).

% Orang yang cerdas adalah orang yang paling banyak mengingat kematian.
%
%-----------------------------------------------------------------------------------------------%

%-----------------------------------------------------------------------------%
\chapter{\babLima}
% -----------------------------------------------------------------------------%
\section{Kesimpulan}
% -----------------------------------------------------------------------------%
Berdasarkan hasil penelitian yang yang dilakukan pada Laboratorium Program Studi Sistem Informasi UIN Suska Riau, maka dapat ditarik kesimpulan yaitu:

\begin{enumerate}
    \item Penelitian ini telah berhasil dalam mengimplementasikan SITARIS menggunakan \textit{Framework} CodeIgniter 4 yang memiliki manfaat signifikan pada Laboratorium Program Studi Sistem Informasi UIN Suska Riau.
    \item Sistem informasi inventaris ini memudahkan pihak Laboratorium dan Program Studi Sistem Informasi dalam pengelolaan barang inventaris secara efektif dan efisien.
\end{enumerate}

% -----------------------------------------------------------------------------%
\section{Saran}
% -----------------------------------------------------------------------------%
Penulis menyadari dalam pelaksanaan KP dan pembuatan laporan maupun sistem masih terdapat celah dan kekurangan. Berdasarkan hal tersebut penulis membuka diri untuk menerima saran maupun kritik yang membangun bagi penulis kedepannya. Adapun saran yang ingin penulis sampaikan diantaranya :
\begin{enumerate}
    \item Pengembangan sistem informasi inventaris laboratorium dengan penambahan fitur, menu, dan perbaikan tampilan serta efisiensi penulisan skrip.
    \item Pengembangan melalui penambahan metode atau algoritma untuk meningkatkan fungsionalitas sistem agar lebih bermanfaat.
    \item Perbaikan detail kecil dalam sistem untuk mengatasi potensi celah kesalahan dan meningkatkan keamanan.
\end{enumerate}