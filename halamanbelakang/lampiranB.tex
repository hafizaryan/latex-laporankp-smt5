%-----------------------------------------------------------------------------------------------%
%
% % Oktober 2022
% Template Latex untuk Laporan Kerja Praktek Program Studi Sistem informasi ini
% Dikembangkan oleh Daffa Takratama Savra (daffatakratama13@gmail.com)

% Template ini dikembangkan dari template yang dibuat oleh Inggih Permana (inggihjava@gmail.com).

% Orang yang cerdas adalah orang yang paling banyak mengingat kematian.
%
%-----------------------------------------------------------------------------------------------%

%-----------------------------------------------------------------------------%
\prefikLampiran{A}

\renewcommand{\thepage}{B - \arabic{page}}
\chapter{Transkip Wawancara atau Hasil Observasi}
%-----------------------------------------------------------------------------%
\begin{flushleft}

  \textbf{TEMA : Proses Pencatatan Barang Pada Laboratorium Sistem Informasi} \\
  \textbf{PENELITI : Hafiz Aryan Siregar} \\
  \textbf{NARASUMBER : Tengku Khairil Ahsyar, S.Kom., M.Kom} \\
  \textbf{JABATAN : Kepala Laboratorium Program Studi Sistem Informasi} \\
  \textbf{LOKASI : Ruang Pusat Penelitian Gedung Baru FST} \\
  \textbf{Hari/tanggal : 04 Agustus 2023}
\end{flushleft}


\begin{flushleft}
  \textbf{Pertanyaan:} Apa tujuan utama laboratorium dalam mengimplementasikan sistem informasi inventaris?

  \textbf{Jawaban:} Tujuan utama adalah meningkatkan efisiensi dalam manajemen inventaris laboratorium, termasuk pelacakan dan pengelolaan peralatan yang dimiliki. Kami ingin mengurangi kerumitan dalam pemantauan inventaris dan memudahkan akses data yang akurat dan real-time.

  \textbf{Pertanyaan:} Apakah laboratorium sudah memiliki sistem informasi inventaris sebelumnya, atau ini akan menjadi pengembangan baru?

  \textbf{Jawaban:} Laboratorium Program Studi Sistem Informasi belum memiliki sistem informasi inventaris sebelumnya, jadi ini akan menjadi pengembangan baru untuk laboratorium ini.

  \textbf{Pertanyaan:} Apa kendala atau masalah utama yang ingin diselesaikan dengan penggunaan sistem informasi inventaris?

  \textbf{Jawaban:} Kami menghadapi masalah seperti kesulitan dalam melacak peralatan yang dipinjam, risiko kehilangan data inventaris, dan kurangnya transparansi dalam penggunaan inventaris. Kami ingin mengatasi masalah ini dengan sistem informasi inventaris.

  \textbf{Pertanyaan:} Apakah laboratorium telah mengidentifikasi kebutuhan spesifik dalam manajemen inventaris yang ingin Anda atasi?

  \textbf{Jawaban:} Ya, kami ingin memiliki kemampuan untuk melacak peminjaman peralatan, dan pemeliharaan rutin peralatan laboratorium.

  \textbf{Pertanyaan:} Bagaimana laboratorium saat ini mengelola dan melacak inventaris laboratorium?

  \textbf{Jawaban:} Saat ini, kami menggunakan spreadsheet manual dan proses manual untuk mencatat dan melacak inventaris laboratorium. Ini kurang efisien dan tidak selalu akurat.

  \textbf{Pertanyaan:} Apakah Anda telah mengidentifikasi tim atau individu yang akan terlibat dalam proyek ini, dan apa peran mereka?

  \textbf{Jawaban:} Ya, kami telah menunjuk tim proyek yang terdiri dari kelompok SIMLAB yaitu tim yang bergerak dalam pengembangan Sistem Informasi Manajemen Laboratorium termasuk Sistem Informasi Inventaris. Masing-masing memiliki peran yang jelas dalam pengembangan dan pelaksanaan proyek.

  \textbf{Pertanyaan:} Apakah Anda telah mempertimbangkan masalah keamanan data dan privasi terkait dengan sistem informasi inventaris?

  \textbf{Jawaban:} Ya, keamanan data adalah prioritas kami. Kami akan menerapkan langkah-langkah keamanan yang diperlukan, termasuk otorisasi akses dan enkripsi data sensitif.

  \textbf{Pertanyaan:} Bagaimana Anda merencanakan pemeliharaan dan dukungan teknis setelah implementasi sistem ini?

  \textbf{Jawaban:} Kami sedang merencanakan dukungan teknis jangka panjang dan pemeliharaan rutin untuk memastikan sistem tetap berjalan dengan baik setelah implementasi.

  \textbf{Pertanyaan:} Bagaimana Anda melihat sistem informasi inventaris laboratorium ini meningkatkan efisiensi operasional dan manajemen inventaris?

  \textbf{Jawaban:} Kami berharap sistem ini akan memungkinkan kami menghemat waktu, mengurangi kesalahan manusia, dan meningkatkan visibilitas atas inventaris kami, yang akan berkontribusi pada efisiensi operasional.

  \textbf{Pertanyaan:} Apakah ada masalah khusus yang perlu Anda selesaikan atau tantangan yang Anda antisipasi dalam proyek ini?

  \textbf{Jawaban:} Kami akan mengukur kesuksesan berdasarkan efisiensi operasional yang meningkat, akurasi data inventaris, dan tingkat kepuasan staf dan pengguna akhir.

\end{flushleft}
